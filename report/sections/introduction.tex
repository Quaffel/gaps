\chapter{Introduction}
The Gaps game is a single-player game played with a deck of French-suited playing cards\footnote{\url{https://en.wikipedia.org/wiki/French-suited_playing_cards}}. It is attributed to the family of patience games\footnote{\url{https://en.wikipedia.org/wiki/Patience_(game)}}, where the objective is to arrange cards in a specific order according to certain rules. Despite its apparent simplicity, Gaps can be time-consuming to solve, especially in its default setting in which all ranks are used.

In Gaps, the cards are arranged in a grid, and the player aims to fill the gaps by moving one card after another. The challenge lies in finding the right sequence of strategic steps to achieve the goal. The complexity of the game increases with the size of the grid, making it a suitable candidate for exploring Artificial Intelligence techniques to find optimal solutions.

\figureWithCaption{gaps}{The playing cards used to play the Gaps game (in its unshuffled state)}{\textwidth}

\section{Terminology}
We employ card game-specific terminology throughout the report:

\begin{enumerate}
    \item \textbf{Gap}: A gap is a cell in the game's grid where no card is present.
    \item \textbf{Rank}: The rank of a card refers to its number, which ranges from Ace (mapped to ID 0) to King (mapped to ID 12).
    A card's ID corresponds to its precedence, with 0 being the lowest precedence.
    \item \textbf{Suit}: The suit of a card represents its symbol: Hearts (mapped to 0), Diamonds (mapped to 1), Clubs (mapped to 2), and Spades (mapped to 3).
    \item \textbf{Color}: The color of a card. 
    Can either be black (club and spades) or red (hearts and diamond).
\end{enumerate}

In addition, we also use the following terms that are specific to the gaps game:

\begin{enumerate}
    \item \textbf{Card Coordinate}: We define coordinates on the board in terms of the grid. A coordinate is denoted as $(r, c)$, where $r$ represents the row and $c$ represents the column. The grid's origin is at the top left. 
    \item \textbf{Previous Card}: The previous card is the card immediately to the left of the current card. Therefore, the previous card of the card at $(r, c)$ is located at $(r, c-1)$.
    \item \textbf{Trap Node / Trap State}: An unsolved state of the game in which no further actions can be played.
    \item \textbf{Solve Node / Solved State}: A state of the game that is considered solved.
    \item \textbf{Leaf Node / Leaf State}: A state that is either a solved state or a trap state.
    \item \textbf{State Space}: A set containing all possible states that can be reached from a given game state (by performing a series of arbitrary actions of arbitrary length).
\end{enumerate}

\section{Mathematical notation}
We will use the following mathematical notation throughout the report:

\begin{table}[ht]
\centering
\resizebox{\columnwidth}{!}{%
\begin{tabular}{ll}
$G^t$           & Represents the board at time $t$. It is a matrix of cards                                                  \\     
$G^t_{c,r}$     & Represents the card at position $(c,r)$ in the board at time $t$                                    \\
$C$           & Number of columns we are working with                                                           \\
$R$           & Number of rows we are working with. This corresponds to the total number of gaps.               \\
$X\left(G^t\right)$ and $X_n\left(G^t\right)$ & Number of well placed cards. The subscript $n$ denotes its normalized version between 0 and 1         \\
$Y\left(G^t\right)$ and $Y_n\left(G^t\right)$ & Number of dead gaps in the game. The subscript $n$ denotes its normalized version between 0 and 1     \\
$Z\left(G^t\right)$ and $Z_n\left(G^t\right)$ & Number of repeated gaps in the game. The subscript $n$ denotes its normalized version between 0 and 1 \\
$S\left(W_X, W_Y, W_Z, G^t\right)$ & Score of the state $G^t$ with the weights $W_X$, $W_Y$, and $W_Z$ \\
$H\left(G^t, W_p\right)$ & Heuristic of the state $G^t$ with a weight penalty $W_p$ \\
$N$           & Complexity of the shuffle\\
$W_c$           & Weight of the exploration. It is a term in the UCB formula, which is used by MCTS\\
$t(C, N)$           & Timeout in seconds for a game of $C$ columns and complexity $N$\\
\end{tabular}%
}
\caption{}
\label{tab:my-table}
\end{table}
\newpage

\section{Game rules}
Gaps is played as follows:

\begin{enumerate}
    \item A game has a number of rows and columns within the discrete intervals\\
    $R \in \{1, 2, 3, 4\}$\\
    $C \in \{3, 4, 5, 6, 7, 8, 9, 10, 11, 12, 13\}$
    \item We create $R$ gaps by removing $R$ (number of rows) from the board. 
    Typically, these are the cards of the highest rank (Kings with 13 columns, Queens with 12, and so on...). 
    Alternatively, it is also possible to remove the cards of the lowest rank (always Ace).
    This does not affect the difficulty of the problem.
    We chose the prior approach.
    \item We randomly shuffle all grid cells, which causes all gaps to end up at random coordinates.
    \item The only possible move is to move a card into a gap.
    This causes a new gap to appear at the card's previous position.
    \item A card can be moved into a gap if and only if the card preceding the gap is of the same suit and has a rank that is exactly one unit lower than the card to be moved. 
    For instance, we can move \MainMiniCartesJeu{7.K} in a gap that is to the right of \MainMiniCartesJeu{6.K}.
    The cards \MainMiniCartesJeu{6.H} (mismatching suit) and \MainMiniCartesJeu{5.K} (rank is two units smaller, not just one) would not be permissible.
    \item Gaps at the beginning of a row can be filled unconditionally.
    That is, any card can be moved into gaps of the set $\{(r, 0) | \forall r \in R\}$.
    \item If a gap immediately follows another, we refer to it as a "repeated gap". Repeated gaps cannot be filled as long as its predecessor has not been filled.
    \item The game is solved if every row is populated with cards of exactly one suit respectively, and if the cards are sorted with respect to their ranks. The gaps have to appear in the last position of a row.
\end{enumerate}

\section{Dead gap corollary}
According to the rules, gaps located after the highest-ranked cards in the game cannot be filled with cards. This is because a card can only be placed in a gap if its rank is exactly one unit higher than the rank of the card preceding the gap. We refer to these unusable gaps as "dead gaps."

Gaps located in the last column are never considered dead, as they are in the correct location with regard to the solved state. Note that they are nevertheless unusable in the outlined scenario.

\figureWithCaption{rules}{The movements rules of the games summarized in one state}{\textwidth}

\section{Shuffling}
There are multiple ways to shuffle board. A naive approach involves selecting two random cards (or gaps) and swapping them, repeating this process $N$ times to shuffle the game. However, this method does not guarantee the game to be solvable. Starting from a solvable game state is typically desired by human players though, as the game is already fairly complex and time-consuming on its own.

A more reliable method is to start from a solved state, then pick a random card and a random gap, and move the card from its initial position to the gap. This process is repeated $N$ times. Shuffling in this manner from a solved game creates a tree of possible actions, with an increasing depth as $N$ increases. The higher the value of $N$, the more trap nodes are introduced into our state space. This is why we refer to the number of swaps as the "complexity of the shuffle" (or the initial board).

\begin{algorithm}[H]
    \caption{Shuffle board}
    \begin{algorithmic}
        \For t from 0 to N
            \State $(r_c, c_c) \gets$ pick\_random\_card\_position$\left(G^t\right)$
            \State $(r_g, c_g) \gets$ pick\_random\_gap$\left(G^t\right)$
            \State $G^t \gets$ swap$\left(G^t_{(r_c,c_c)}, G^t_{(r_g,c_g)}\right)$
        \EndFor
    \end{algorithmic}
\end{algorithm}
