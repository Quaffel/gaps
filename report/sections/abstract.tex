\chapter{Abstract}
Gaps, also known as Montana Gaps or Patience Gaps\footnote{\url{https://en.wikipedia.org/wiki/Gaps}}, is a solitaire game played with a standard 52-card deck of French-suited playing cards, known for being time-consuming to solve in its default setting. The goal is to arrange cards in a specific order according to certain rules. Despite its popularity among solitaire games, there has been no significant work dedicated to solving this game algorithmically, and we aim to contribute to this area through our project.

The three algorithms we implemented are Greedy BFS, A$^*$, and Monte Carlo Tree Search (MCTS). Greedy BFS is a heuristic-based approach that prioritizes reaching a leaf node quickly without necessarily finding the shortest path. A$^*$, covered in class, is a well-known pathfinding algorithm that focuses on exploring nodes with a low cost and a low estimated cost first, aiming to find the shortest path. MCTS, which we explored independently, is a stochastic algorithm that combines exploration and exploitation to gradually converge on an optimal solution through repeated simulations.

We conducted experiments across various board sizes and complexities, measuring the performance of each algorithm in terms of time taken to solve the game, path length to a possible solution, and success rate. Our analysis provides insights into the suitability of each algorithm for solving the Gaps game, highlighting their respective strengths and weaknesses.

In this report, we detail the implementation and evaluation of these algorithms, discuss their performance in solving the Gaps game, and suggest potential improvements and ideas for future research.
